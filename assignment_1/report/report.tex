\documentclass{article}
\usepackage{graphicx}
\usepackage{listings}
\usepackage{hyperref}
\usepackage{float}

\title{Assignment for CS 696-04 \\ (Deep Learning)}
\author{Diwas Sharma}
\date{\today}
\begin{document}

\maketitle
\newpage

\section{Design}
The code that is used to construct and train the Neural network is highly modular and
can be used as a library for other programs. The interface is quite similar to the one
provided by keras \footnote{\url{https://keras.io/}}.

The code enable the user to construct a neural network with any number of dense
layers. And, supports activation functions such as sigmoid, hyperbolic tangent,
ReLu, Leaky ReLu, and linear. An example of a neural network that can be construct
using the code is given below.

\begin{lstlisting}[language=python]
from depth.sequential import NeuralNet

nn_object = NeuralNet()
nn_object.add_layer(units=32, activation_function="relu",
                    input_dimension=10)
nn_object.add_layer(units=32, activation_function="tanh")
nn_object.add_layer(units=32, activation_function="sigmoid")
\end{lstlisting}

It also provides a choice between cross entropy loss and mean squared
error to be used as a loss function for the output layer.

\section{MNIST dataset}
\subsection{Network}
The network consists of 3 hidden layers each with 32 neurons.  The first and the last
hidden layer uses hyperbolic tangent as the activation function. And the second
hidden layer uses the ReLu function as the activation function. The output layer
is a hyperbolic tangent layer with 10 neurons with softmax output.

As the MNIST dataset is a multiclass classification problem, the cross entropy function
is used as the loss function along with gradient descent optimization for adjusting
parameters.

\subsubsection{Source Code}
\begin{lstlisting}[language=python]
self.nn = NeuralNet()
self.nn.add_layer(units=32, activation_function="tanh",
                    input_dimension=self.input_data_dimension)
self.nn.add_layer(units=32, activation_function="relu")
self.nn.add_layer(units=32, activation_function="tanh")
self.nn.add_layer(units=self.output_data_dimension,
                    activation_function="tanh")
self.nn.compile(loss="cross_entropy",
                error_threshold=self.error_threshold,
                optimizer=optimizer)
\end{lstlisting}

\subsection{Training}
At first a learning rate of $\eta = 0.001$ was chosen as a starting point
to observe the performance of the network. The training was done using gradient
descent algorithm with momentum where momentum coefficient$(\gamma) = 0.9$.  The plot in figure \ref{fig:mnist_three_loss} shows
the regularized loss function at the end of each epoch during the training. Similarly, figure \ref{fig:mnist_three_accuracy} shows the
training accuracy at the end of every epoch.

\begin{figure}[!ht]
  \includegraphics[width=\textwidth,height=0.35\textheight,keepaspectratio]{m1e-3_m_loss.png}
  \caption{Regularized training loss for $\eta = 0.001, \gamma = 0.9$}
  \label{fig:mnist_three_loss}
\end{figure}

\begin{figure}[!ht]
  \includegraphics[width=\textwidth,height=0.35\textheight,keepaspectratio]{m1e-3_m_accuracy.png}
  \caption{Training accuracy for $\eta = 0.001$}
  \label{fig:mnist_three_accuracy}
\end{figure}

It is clear from the plots that the learning rate of $0.001$ is too small for training the network. A
significant improvement can be seen in the training performance if the learning rate is increased to $\eta = 0.5$.
The performance of network for the increased learning rate is show in figures \ref{fig:mnist_one_loss} and \ref{fig:mnist_one_accuracy}.

\begin{figure}[!ht]
  \includegraphics[width=\textwidth,height=0.35\textheight,keepaspectratio]{m5e-1_m_loss.png}
  \caption{Regularized training loss for $\eta = 0.5, \gamma = 0.9$}
  \label{fig:mnist_one_loss}
\end{figure}

\begin{figure}[!ht]
  \includegraphics[width=\textwidth,height=0.35\textheight,keepaspectratio]{m5e-1_m_accuracy.png}
  \caption{Training accuracy for $\eta = 0.5$}
  \label{fig:mnist_one_accuracy}
\end{figure}

\subsection{Testing}
The accuracy obtained for the network on the test data is 0.89403.

\section {CIFAR-10 dataset}
\subsection{Network}
The network consists of 3 hidden layers each with 128 neurons.  The first and the last
hidden layer uses hyperbolic tangent as the activation function. And the second
hidden layer uses the ReLu function as the activation function. The output layer is
a hyperbolic tangent layer with 10 neurons with softmax output.

The loss function used is cross entropy loss and gradient descent is used for
optimizing the parameters.

\subsubsection{Source Code}
\begin{lstlisting}[language=python]
self.nn = NeuralNet()
self.nn = NeuralNet()
self.nn.add_layer(units=128, activation_function="tanh",
                    input_dimension=self.input_data_dimension)
self.nn.add_layer(units=128, activation_function="relu")
self.nn.add_layer(units=128, activation_function="tanh")
self.nn.add_layer(units=self.output_data_dimension,
                    activation_function="tanh")
self.nn.compile(loss="cross_entropy",
                error_threshold=self.error_threshold,
                optimizer=optimizer)
\end{lstlisting}

\subsection{Training}
The performance of the network for a learning rate of $\eta = 0.001$ is shown in figures
\ref{fig:cifar_three_loss} and \ref{fig:cifar_three_accuracy}.

\begin{figure}[!ht]
  \includegraphics[width=\textwidth,height=0.35\textheight,keepaspectratio]{c1e-3_loss.png}
  \caption{Training loss for $\eta = 0.001$}
  \label{fig:cifar_three_loss}
\end{figure}

\begin{figure}[!ht]
  \includegraphics[width=\textwidth,height=0.35\textheight,keepaspectratio]{c1e-3_accuracy.png}
  \caption{Training accuracy for $\eta = 0.001$}
  \label{fig:cifar_three_accuracy}
\end{figure}

Increasing the learning rate to $\eta = 0.5$ yields the training performance shown in figures
\ref{fig:cifar_one_loss} and \ref{fig:cifar_one_accuracy}

\begin{figure}[!ht]
  \includegraphics[width=\textwidth,height=0.35\textheight,keepaspectratio]{c5e-1_loss.png}
  \caption{Training loss for $\eta = 0.5$}
  \label{fig:cifar_one_loss}
\end{figure}

\begin{figure}[!ht]
  \includegraphics[width=\textwidth,height=0.35\textheight,keepaspectratio]{c5e-1_accuracy.png}
  \caption{Training accuracy for $\eta = 0.5$}
  \label{fig:cifar_one_accuracy}
\end{figure}

\subsection{Testing}
The accuracy obtained for the network on the test data is 0.282.

\end{document}
