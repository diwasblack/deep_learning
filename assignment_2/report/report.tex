\documentclass{article}
\usepackage{graphicx}
\usepackage{listings}
\usepackage{hyperref}
\usepackage{float}

\title{Assignment for CS 696-04 \\ (Deep Learning)}
\author{Diwas Sharma}
\date{\today}
\begin{document}

\maketitle
\newpage

\section{Design}
The implementation of convolution and max pooling layer is generalized enough
so it can be used with other programs as well. Furthermore, the code for the
CNN layers have been merge with the implementation of the Dense layers.
Together the dense layers and the CNN layers can be stacked to any depth required
and can be used to describe any valid neural network with dense and CNN layers.

An example of convolution neural network that can be construct is given below.

\begin{lstlisting}[language=python]
from depth.models import Sequential
from depth.layers import Convolution2D, Flatten, DenseLayer, MaxPooling
from depth.optimizers import ADAM

optimizer = ADAM(lr=0.01)
nn = Sequential()

self.nn = Sequential()
self.nn.add_layer(Convolution2D(5, (3, 3), input_shape=(3, 32, 32)))
self.nn.add_layer(MaxPooling(pool_size=(2, 2)))
self.nn.add_layer(Convolution2D(10, (3, 3)))
self.nn.add_layer(MaxPooling(pool_size=(2, 2)))
self.nn.add_layer(Flatten())
self.nn.add_layer(DenseLayer(units=32))
self.nn.add_layer(DenseLayer(units=10, activation="softmax"))
self.nn.compile(loss="cross_entropy", error_threshold=0.01,
                optimizer=optimizer)

\end{lstlisting}

\section{CIFAR 10}
\subsection{Network}
The network that is used to classify the CIFAR 10 dataset is follows:

\begin{enumerate}
    \item{Convolution layer}
        \begin{itemize}
            \item{Input: $N * 3 * 32 * 32$ tensor}
            \item{Filters: 4 filters with size $3*3$}
            \item{Padding: 0 padded with pad width of 1}
            \item{Stride: 1 on each dimension}
            \item{Regularizer: $L_2$ Regularizer}
        \end{itemize}
    \item{Max pooling layer}
        \begin{itemize}
            \item{Input: $N * 4 * 32 * 32$ tensor}
            \item{Pool size: $2*2$}
            \item{Stride: 2 on each dimension}
        \end{itemize}
    \item{Convolution layer}
        \begin{itemize}
            \item{Input: $N * 4 * 16 * 16$ tensor}
            \item{Filters: 8 filters with size $3*3$}
            \item{Padding: 0 padded with pad width of 1}
            \item{Stride: 1 on each dimension}
            \item{Regularizer: $L_2$ Regularizer}
        \end{itemize}
    \item{Max pooling layer}
        \begin{itemize}
            \item{Input: $N * 8 * 16 * 16$ tensor}
            \item{Pool size: $2*2$}
            \item{Stride: 2 on each dimension}
        \end{itemize}
    \item{Convolution layer}
        \begin{itemize}
            \item{Input: $N * 8 * 8 * 8$ tensor}
            \item{Filters: 16 filters with size $3*3$}
            \item{Padding: 0 padded with pad width of 1}
            \item{Stride: 1 on each dimension}
            \item{Regularizer: $L_2$ Regularizer}
        \end{itemize}
    \item{Softmax layer}
        \begin{itemize}
            \item{Input: $512 * N$ tensor}
            \item{Units: 10}
        \end{itemize}
\end{enumerate}

\subsubsection{Source Code}
\begin{lstlisting}[language=python]
optimizer = ADAM(lr=0.01)
regularizer = L2Regularizer(0.01)

self.nn = Sequential()
self.nn.add_layer(Convolution2D(
    4, (3, 3), input_shape=(3, 32, 32), regularizer=regularizer))
self.nn.add_layer(MaxPooling(pool_size=(2, 2)))
self.nn.add_layer(Convolution2D(8, (3, 3), regularizer=regularizer))
self.nn.add_layer(MaxPooling(pool_size=(2, 2)))
self.nn.add_layer(Convolution2D(16, (3, 3), regularizer=regularizer))
self.nn.add_layer(Flatten())
self.nn.add_layer(DenseLayer(units=10, activation="softmax"))
self.nn.compile(loss="cross_entropy", error_threshold=0.01,
                optimizer=optimizer)
\end{lstlisting}

\subsection{Training}
The network was trained using ADAM optimizer with mini batch size of 256. The parameters
used for the optimizer were $\eta$ = 0.01, $\beta_1$ = 0.9 and 
$\beta_2$ = 0.999.

The plot for training loss per mini batch is shown in figure {\ref{fig:training_loss}}
and the plot of training accuracy per mini batch is shown in figure {\ref{fig:training_accuracy}}.

\begin{figure}[!ht]
  \includegraphics[width=\textwidth,height=0.35\textheight,keepaspectratio]{cifar10_training_loss.png}
  \caption{Training Loss}
  \label{fig:training_loss}
\end{figure}

\begin{figure}[!ht]
  \includegraphics[width=\textwidth,height=0.35\textheight,keepaspectratio]{cifar10_training_accuracy.png}
  \caption{Training accuracy}
  \label{fig:training_accuracy}
\end{figure}

\subsection{Testing}
The accuracy obtained on the test data is 0.536.

\end{document}
