\documentclass{article}
\usepackage{graphicx}
\usepackage{listings}
\usepackage{hyperref}
\usepackage{float}

\title{Assignment for CS 696-04 \\ (Deep Learning)}
\author{Diwas Sharma}
\date{\today}
\begin{document}

\maketitle
\newpage

\section{Design}
The code that is used to train and test CIFAR-10 dataset uses the interface provided by
keras \footnote{https://keras.io/}.

\section{CIFAR 10}
\subsection{Network}
The network that is used to classify the CIFAR 10 dataset is follows:

\begin{enumerate}
    \item{Convolution layer}
        \begin{itemize}
            \item{Input: $N * 3 * 32 * 32$ tensor}
            \item{Filters: 4 filters with size $3*3$}
            \item{Padding: 0 padded with pad width of 1}
            \item{Stride: 1 on each dimension}
            \item{Regularizer: $L_2$ Regularizer}
        \end{itemize}
    \item{Max pooling layer}
        \begin{itemize}
            \item{Input: $N * 4 * 32 * 32$ tensor}
            \item{Pool size: $2*2$}
            \item{Stride: 2 on each dimension}
        \end{itemize}
    \item{Convolution layer}
        \begin{itemize}
            \item{Input: $N * 4 * 16 * 16$ tensor}
            \item{Filters: 8 filters with size $3*3$}
            \item{Padding: 0 padded with pad width of 1}
            \item{Stride: 1 on each dimension}
            \item{Regularizer: $L_2$ Regularizer}
        \end{itemize}
    \item{Max pooling layer}
        \begin{itemize}
            \item{Input: $N * 8 * 16 * 16$ tensor}
            \item{Pool size: $2*2$}
            \item{Stride: 2 on each dimension}
        \end{itemize}
    \item{Convolution layer}
        \begin{itemize}
            \item{Input: $N * 8 * 8 * 8$ tensor}
            \item{Filters: 16 filters with size $3*3$}
            \item{Padding: 0 padded with pad width of 1}
            \item{Stride: 1 on each dimension}
            \item{Regularizer: $L_2$ Regularizer}
        \end{itemize}
    \item{Softmax layer}
        \begin{itemize}
            \item{Input: $512 * N$ tensor}
            \item{Units: 10}
        \end{itemize}
\end{enumerate}

\subsubsection{Source Code}
\begin{lstlisting}[language=python]
# Create the model
model = Sequential()

# Create the regularizer
regularizer = l2(0.01)

model.add(
    Conv2D(
        4, kernel_size=(3, 3), activation='relu', input_shape=(32, 32, 3),
        kernel_regularizer=regularizer
    )
)
model.add(MaxPooling2D(pool_size=(2, 2)))
model.add(
    Conv2D(
        16, kernel_size=(3, 3), activation='relu',
        kernel_regularizer=regularizer
    )
)
model.add(MaxPooling2D(pool_size=(2, 2)))
model.add(
    Conv2D(
        32, kernel_size=(3, 3), activation='relu',
        kernel_regularizer=regularizer
    )
)
model.add(Flatten())
model.add(Dense(10, activation='softmax'))

# Compile the model
model.compile(loss='categorical_crossentropy',
              optimizer=Adam(lr=0.0001, decay=1e-6),
metrics=['accuracy'])
\end{lstlisting}

\subsection{Training}
The network was trained using ADAM optimizer with mini batch size of 128. And, the
parameters used for the optimizer were $\eta$ = 0.0001, $\beta_1$ = 0.9 and
$\beta_2$ = 0.999 with learning rate decay of $1e-6$.

The figure \ref{fig:training_loss} shows the plot of loss at the end of each epoch
during the training, similarly the figure \ref{fig:training_accuracy} shows the
accuracy at the end of each epoch during the training phase.

\begin{figure}[!ht]
  \includegraphics[width=\textwidth,height=0.35\textheight,keepaspectratio]{cifar10_loss.png}
  \caption{Training loss}
  \label{fig:training_loss}
\end{figure}

\begin{figure}[!ht]
  \includegraphics[width=\textwidth,height=0.35\textheight,keepaspectratio]{cifar10_accuracy.png}
  \caption{Training accuracy}
  \label{fig:training_accuracy}
\end{figure}

\subsection{Testing}
The accuracy obtained on the test data after training for 300 epochs is 0.6074.

\end{document}
